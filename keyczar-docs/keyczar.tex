
\documentclass{llncs}

\title{Keyczar: A Cryptographic Toolkit}
\author{Arkajit Dey\inst{1} and Stephen Weis\inst{2}}

\institute{Stanford University, Stanford, CA, USA 94305
\and
Google Inc., Mountain View, CA, USA 94043}

\begin{document}
\maketitle

\begin{abstract}
Keyczar's goal is to make it easier for application developers to safely use
cryptography. Keyczar defaults to safe algorithms, key lengths, and
modes, and prevents developers from inadvertently exposing key material. It
uses a simple, extensible key versioning system that allows developers to
easily rotate and retire keys. 
\end{abstract}

\section{Introduction and Philosophy}

The motivation for Keyczar grew out of a need to make cryptography easier to
use for developers. Developers often can make simple mistakes when using
cryptography that can create security vulnerabilities. For instance, developers
may use obsolete algorithms, weak key lengths, improper cipher modes, or
unsafely compose cryptographic operations. Another common developer mistake is
to fail to provision for key rotation or even to hard-code keys in source
code.

Keyczar's goal is to address these issues by providing a simple application
programming interface (API) for developers that handles basic cryptographic
details. Keyczar also provides a simple key versioning and management system
based on directories of human-readable flat files, which will be refered to as
{\it keysets}. 
\section{Using KeyczarTool}

All Keyczar keys are generated with the stand-alone
{\tt KeyczarTool} utility located in the {\tt com.google.keyczar} Java package.

\subsection{KeyczarTool create}
KeyczarTool must first create a new keyset using the {\tt create} command. A
newly created keyset will initially contain just a metadata file, described in
section \ref{metadata}.

{\tt KeyczarTool create} requires {\tt location} and {\tt purpose} command-line
flags that specify the location of the key set and its purpose. Valid purposes
are currently {\tt crypt} and {\tt sign}. The create command may also take
an optional {\tt name} flag to give a newly created keyset a name. If the {\tt
asymmetric} flag is specified, the newly created set will contain asymmetric
keys.

Some example {\tt create} commands:
\begin{itemize}
\item Create a symmetric signing (HMAC) keyset: \\
{\tt KeyczarTool create --location=/path/to/keyset --purpose=sign}
\item Create a symmetric crypting (AES) keyset named ``Test'': \\
{\tt KeyczarTool create --location=/path/to/keyset --purpose=crypt --name=Test}
\item Create an asymmetric signing (DSA) keyset: \\
{\tt KeyczarTool create --location=/path/to/keyset --purpose=sign --asymmetric} 
\end{itemize}

\subsection{KeyczarTool addkey}

All Keyczar keys are created using the {\tt addkey} command. This command
requites a keyset {\tt location} flag and may optionally have {\tt status} and
{\tt size} flags. Section \ref{status} describes the meaning of the status
values, but briefly they are {\it primary}, {\it active}, and {\it
scheduled\_for\_revocation}. The default status is {\it active}. User-specified
key sizes are supported, although it is recommneded that on default or larger
key  sizes are used.

The {\tt addkey} command will create a new file in the keyset directory with an
integer version number that is one greater than the currently largest version.
Version numbers start from 1 and are described in Section \ref{versions}. For
example, if the current keyset contains the key file {\tt 1}, a new key version
will be created in the file {\tt 2}. Some example {\tt addkey} commands:
\begin{itemize}
\item Create a new primary key: \\
{\tt KeyczarTool addkey --location=/path/to/keyset --status=primary}
\item Create a new active key: \\
{\tt KeyczarTool addkey --location=/path/to/keyset} 
\end{itemize}

\subsection{KeyczarTool pubkey}

Public keys in Keyczar are exported from existing asymmetric key sets. The
{\tt pubkey} command requires both an existing {\tt location} flag and a {\tt
destination} where public keys will be exported. If the keyset under {\tt
location} was not created with an {\tt asymmetric} flag, then a {\tt pubkey}
command will fail. An example {\tt pubkey} command works as follows: \\
{\tt KeyczarTool pubkey --location=/path/to/keyset --destination=/path/to/dest}

\subsection{KeyczarTool promote, demote, and revoke}

The {\tt promote}, {\tt demote}, and {\tt revoke} commands are used to change
key status values. Each of these commands require a {\tt location} and {\tt
version} flag. 

Promoting an {\it active} key will raise its status to
{\it primary}, and promoting a {\it scheduled\_for\_revocation} status will make
it {\it primary}. There can only be a single {\it primary} key in a given key
set. 

Similarly, {\tt demote} will lower a {\it primary} key to {\it active},
and an {\it active} key to {\it scheduled\_for\_revocation}. The {\tt revoke}
command will only work for {\it scheduled\_for\_revocation} status values.
The {\tt revoke} command will permenantly delete key material, so should be
used with caution.

Some example {\tt promote}, {\tt demote}, and {\tt revoke} commands. Suppose
that key version 1 initially has an {\it active} status:
\begin{itemize}
\item Promote {\it active} version 1 to {\it primary}: \\
{\tt KeyczarTool promote --location=/path/to/keyset --version=1}
\item Demote {\it primary} version 1 back to {\it active}: \\
{\tt KeyczarTool demote --location=/path/to/keyset --version=1}
\item Demote {\it active} version 1 to {\it scheduled\_for\_revocation}: \\
{\tt KeyczarTool demote --location=/path/to/keyset --version=1}
\item Revoke the {\it scheduled\_for\_revocation} version 1:  
\item {\tt KeyczarTool revoke --location=/path/to/keyset --version=1}
\end{itemize}




\section{Using Java Keyczar}

\section{Using Python Keyczar}

\section{Keys}
\subsection{Key Metadata}\label{metadata}
\subsection{Identifying Headers}
\subsection{Versions}\label{versions}
\subsection{Statuses}\label{status}
\subsection{Formats}
\subsubsection{HMAC}
\subsubsection{AES}
\subsubsection{RSA}
\subsubsection{DSA}



\section{Output Formats}
\subsection{Ciphertext}
\subsection{Signatures}

\section{Licenses and Dependencies}

Keyczar is available under an Apache 2.0 license \cite{apache2}. Java Keyczar
depends on the Google GSON package \cite{google-gson}. It also relies on the
Java's {\tt javax.crypto} package, which may not be available in  all locales
due to local laws and regulations.

Python Keyczar depends on the Python Cryptography Toolkit \cite{python-crypto}
and simplejson \cite{simplejson}. 

\section{Acknowledgements}

Thanks to Ben Laurie, Marius Schilder, and Neil Daswani for their
contributions.

\bibliography{keyczar}
\bibliographystyle{acm}

\end{document}
